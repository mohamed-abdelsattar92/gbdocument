``Time is money'' \footnote{The phrase is usually credited to Benjamin Franklin, who used it in an essay \textit{(Advice to a Young Tradesman, 1748).}}, the most important aspect of service companies is good response time, we don't want a broken asset to be fixed with the most perfect quality but after 10 days of reporting the problem, do we ?. 

Usually good response time means a happier customer and a more prosperous business, bad response time almost always anger the customers and raise the probability of losing them to competitors, companies that are'nt up to the challenge face serious problems.

Inventory assets tracking is another issue, many service companies have problems in tracking their spare parts in the inventory and in the field, which spare part is taken by which technician, billing and invoicing using pencil and paper can result in many problems and forgery, hence money loss and unhappy customer.

Pencil and paper can work with tens of customers but once we're dealing with hundreds, thousands or hundreds of thousands of customers we need a better approach to tackle the problems of dispatching, scheduling, routing, inventory assets tracking and technician tracking. That's why Field Service Maestro is the best solution for such companies that have a moderate to semi-large work load.

\subsection{Educational Value}
	In this project we acquired a lot of skills some technical some non-technical :
	\subsubsection{Technical Skills} 
		\begin{itemize}
			\item Software architecture and design.
			\item Algorithms design.
			\item Understanding the client-server approach.
			\item Web services design and implementation.
			\item Android development.
			\item Software engineering methodologies.
		\end{itemize} 
	\subsubsection{Non-Technical Skills}
		\begin{itemize}
			\item Understanding the business process of large service companies such as Zanusi, B-Tech and Toshiba.
			\item How to present our work to get sponsorship and support from companies.
			\item Time management and some project management skills.
			\item How to write good documentation for a software project.
		\end{itemize}