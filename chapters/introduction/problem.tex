Field service has traditionally seemed a quite simple process. An asset or a piece of equipment breaks, the customer notifies the service company, the dispatcher schedules a field technician to the task, the technician arrives within an agreed upon service window, fixes the asset, and then moves on to the next job.

This process repeats itself over and over for many service organizations on a daily basis, thus service companies need a way to help them manage dispatching, scheduling, inventory assets tracking and technicians tracking. Sticky notes, paperwork and mobile telephony have managed to drive the work this far.

The previous approach works fine for small businesses, however for medium and large companies sticky notes won't do the trick, after the first thousand assignments and the 100\textsuperscript{th} technician being hired things start to grow out of their hands, dispatching becomes a nightmare, companies start to break their service level agreement (SLA) \footnote{Service level agreement is a part of a standardized service contract where a service is formally defined. Particular aspects of the service – scope, quality, responsibilities – are agreed between the service provider and the service user.} and eventually creating an unfavorable brand image and start losing customers for competitors, so a more automated solution is needed.
