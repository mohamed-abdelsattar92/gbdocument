Character sets and encoding are two important concepts for any program or web application that needs to use languages other than the English language.

\subsubsection{Character Sets}
	A character set is just a set of characters related somehow to each other, the alphabet for example is the English character set.
	A coded character set is a character set where each character is represented by a number (usually in hexadecimal format) called a code point\cite{char_enc}, for example the English character set can be represented by the ASCII\footnote{American Standard Code for Information Interchange.}.
	
	``Unicode is a universal character set, a standard that defines, in one place, all the characters needed for writing the majority of living languages in use on computers. It aims to be, and to a large extent already is, a superset of all other character sets that have been encoded.''\cite{char_enc}]
	
\subsubsection{Encoding}
	It is the mapping between code points and bytes in computer memory. There're many encoding schemes like UTF-8\footnote{Unicode Transformation Format.}, UTF-16, UTF-32, and ISO 8859-6\footnote{Arabic language encoding.}, however UTF-8 is the most widely used, and in fact we use it with the Unicode in our product.