\begin{flushright}
	\textarabic[utf]{الخدمة الميدانية هي عملية تحدث في العديد من المؤسسات الخدمية حيث يقوم العملاء باعلام المؤسسة الخدمية بشكوي معينة وتقوم المؤسسة بتسجيل الشكاوي والاتفاق مع فني ميداني معين  لديها وإبلاغ العملاء بالموعد المتفق عليه والقيام بالمتابعة معهم حتي يتم الانتهاء من حل الشكاوي.}
\end{flushright}

\begin{flushright}
	\textarabic[utf]{تتكرر هذه العملية بشكل يومي لدي الكثير من المؤسسات الخدمية  مما ينتج عنه إهدار الوقت والمال حيث تلجأ المؤسسات الي إتباع الطرق التقليديه كالمستندات الورقية.}
\end{flushright}

\begin{flushright}
	\textarabic[utf]{يقوم فيلد سيرفس مايسترو بتنظيم الخدمة الميدانية بطريقة أوتوماتيكية بدون اللجوء الي الطرق التقليدية السابقة مما يؤدي الي تنظيم الوقت و الجهد و توفير المال للمؤسسات و إرضاء العملاء.}
\end{flushright}

\begin{flushright}
	\textarabic[utf]{ينقسم فيلد سيرفس مايسترو الي جزئين رئيسيين :}
\end{flushright}

\begin{flushright}
	\textarabic[utf]{الأول هو الويب و هو  يحتوي علي معلومات كافية عن الفنيين الميدانيين والعملاء والشكاوي و قطع الغيار المتاحة ويكون المسؤول عنه هو الموزع حيث يقوم بتوزيع المهام علي الفنين الميدانين بسهولة وسرعة دون الحاجة إلي آي مستندات ورقية ومتابعة خط سير الشكوي حتي يتم التأكد من أنه تم التعامل معها وحلها .}
\end{flushright}

\begin{flushright}
	\textarabic[utf]{الثاني هو الأندرويد و هو  يحتوي علي الزيارات الخاصة لكل فني ويكون المسؤول عنه هو الفني حيث يستطيع معرفة الزيارات المكلف بها دون الحاجة لزيارة المؤسسة الخدمية بشكل يومي و يستطيع الموزع معرفة مكان الفني و التأكد من انه متواجد في الأماكن المكلف بالذهاب إليها و أنه يباشر عمله بالفعل كما يستطيع أيضا معرفة قطع الغيار التي تم إستخدامها أثناء الزيارات المختلفة لضمان عدم التلاعب في الأسعار وضمان إتمام العملية بصورة أكثر نزاهة .}
\end{flushright}
